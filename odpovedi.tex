% -------------------------------------------------------
% Úkol 1:
% -------------------------------------------------------
Při jakém prahovém napětí začne diodou proudit proud?\\
	Měřitelný proud začal diodou proudit při namětí cca $(2,00\pm 0,05)\unit{V}$. \\

Co se stane při postupném zvyšování napětí? Při jakém napětí a proudu dojde ke zničení diody? \\
	Dioda bude svítit stále jasněji, při příliš vysokém napětí začne její světlo z důvodu přehřívání červenat a slábnout. \\
	Ke zničení diody došlo při napětí $(6,20\pm 0,05)\unit{V}$ a proudu $(0,160\pm 0,005)\unit{A}$. \\

Srovnejte výsledky s parametry uváděnými výrobcem na přiloženém letáku. Takto získáte představu o tom, jaký proud je pro diodu krákodobě snesitelný. \\
	Dle výrobce bylo maximální stabilně snesitelné napětí pro diodu $(2,80\pm 0,05)\unit{V}$, což podle naměřených dat odpovídá proudu $(0,040\pm 0,005)\unit{A}$. \\
	Hodnota nejvyššího povoleného proudu byla pro tuto diodu výrobcem stanovena jako 0,15 A, což je blízké hodnotě proudu, při kterém došlo k nenávratnému poškození naší měřené diody. Aby diodou protékal takovýto proud, museli jsme napětí na zdroji zvýšit na $(6,10\pm 0,05)\unit{V}$. \\
	
Proveďte obdobné měření pro ostatní diody
	Jiné diody jsme k dispozici pro naše měření neměli.
	
% -------------------------------------------------------
% Úkol 2:
% -------------------------------------------------------
Odpojte zdroj a sestavte obvod pro měření napětí a proudu dle schématu. % \ref{}
V analogii s předchozím úkolem použijte ochranný odpor $R$. \\

Jakou zvolíte velikost odporu a proč? \\
	Zvolili jsme odpor o velikosti cca 1 $\unit{k\Omega}$, protože byl jediný, který byl pro měření k dispozici. Tato velikost odporu se vzhledem k přesnosti našeho měření ukázala být příliš vysoká -- měřili jsme proud s přesností na 0,005 A, ale i při maximálním napětí zdroje jsme naměřili pouze 0,02 A procházející diodou. \\
	
Změřte V-A charakteristiku v propustném směru pro alespoň tři diody. \\
	Vzhledem k malé změně proudu na diodě při tomto měření nebylo možné z naměřených dat sestavit graf V-A charakteristiky. Použili jsme proto data z prvního měření, při kterém do obvodu nebyl zapojen odpor. Více diod pro měření jsme dispozici neměli. \\
